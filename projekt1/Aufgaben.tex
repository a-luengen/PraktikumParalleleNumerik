\documentclass{article}
\usepackage[utf8]{inputenc}


\title{Antworten zu Aufgabenblatt 1}
\date{2019-06-04}
\author{Alexander Lüngen}

\begin{document}
    \maketitle
    \newpage
    \section{Aufgabe 1}
        \subsection{a)}
        Es ist folgendes zu beobachte:
        \subsection{b)}

        \subsection{c)}

    \section{Aufgabe 2}

    \section{Aufgabe 3}
        \subsection{a)}
        Speedup:
        Effizienz:
        \subsection{b)}
        Race-Conditions: Wenn zwei Threads unabhängig voneinander auf eine Ressource lesend oder auch schreibend zugreiffen können, 
        spricht man von einer Race-Condition. Hierbei kann es bspw. beim Zugriff auf Variablen bei ungünstiger Ausführungszeiten
        dazu kommen, dass am Ende der Berechnungen ein falscher Wert in der Variable enthalten ist, als wenn die Berechnung sequentiell 
        ausgeführt worden wäre.
        Um eine Race-Condition zu vermeiden, können die kritischen Abschnitte in der Art und weise gesichert werden, das immer nur ein 
        Thread gleichzeitig innerhalb des kritischen Abschnitts sein darf.

        In diesem Zusammenhang kann es auch zu Deadlocks, Lifelocks oder auch Starvation kommen. 
        \subsection{c)}
        CPU
        GPGPU
        FPGA
        MICs

    \section{Aufgabe 4}

    \section{Aufgabe 5}
\end{document}